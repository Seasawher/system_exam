\documentclass{jsarticle}
\usepackage{amsmath,amssymb,amsfonts}
\begin{document}

\title{系登録試験 数学}
\author{s2s}
\date{}
\maketitle

\section*{2016年問題}
\section*{問1}
\noindent
$\alpha \in \mathbb{C}$に対して行列$A(\alpha)$を
$$
A(\alpha) = \begin{pmatrix} 1-2\alpha &4\alpha &1-3\alpha \\ -\alpha &1 + 2\alpha & 1-2\alpha \\ 0 & 0 &1 \end{pmatrix}
$$
で定める。$A(\alpha)$と$A(\beta)$が相似になるのはいつか、決定せよ。\\ \\

\section*{問2}
\noindent
次の積分の値を求めよ。
$$
\int_{D} |x|^3 e^{- y^3}dxdy \ \ \ \ \ \ D =\{(x, y) \in \mathbb{R}^2 \mid 0 \leq y \leq 1, \ x^2 \leq y \}
$$ \\ \\

\section*{問3}
$\mathbb{R}^5$の元$a_1, \ldots , a_6$を
$$
a_{1} = \begin{pmatrix} 1 \\ 0 \\ 1 \\ 0 \\ 1 \end{pmatrix} \ \
a_{2} = \begin{pmatrix} 0 \\ 1 \\ 0 \\ 1 \\ 0 \end{pmatrix} \ \
a_{3} = \begin{pmatrix} 1 \\ 1 \\ 0 \\ 0 \\ 1 \end{pmatrix} \ \
a_{4} = \begin{pmatrix} 1 \\ 1 \\ 0 \\ 1 \\ 1 \end{pmatrix} \ \
a_{5} = \begin{pmatrix} 0 \\ 1 \\ 0 \\ 1 \\ 0 \end{pmatrix} \ \
a_{6} = \begin{pmatrix} 1 \\ 0 \\ 0 \\ 0 \\ 1 \end{pmatrix} \ \
$$
により定める。$\mathbb{R}^5$の部分空間$W_1$, $W_2$を$W_1 = < a_1, a_2, a_3 >$, $W_2 = < a_4, a_5, a_6 >$と定める。\\
(1) $\dim W_1$と$\dim W_2$を求めよ。\\
(2) $\dim(W_1+W_2)$を求めよ。\\
(3) $\dim(W_1 \cap W_2)$を求めよ。\\ \\

\section*{問4}
\noindent
$\alpha > 0$とする。$n \in \mathbb{Z}_{>0}$に対して$f_n : [ 0, \infty) \rightarrow \mathbb{R}$を
$$
f_n(x) = \frac{nx^2}{(1 + nx)^{\alpha}}
$$
と定める。\\
(1) $\{ f_n(x)\}$が区間$[0, \infty)$上で0に各点収束するための$\alpha$の条件は何か。\\
(2) $\{ f_n\}$が区間$[0, \infty)$上で0に一様収束するための$\alpha$の条件は何か。\\ \\

\newpage
\section*{2016年解答例}
\section*{問1}
特性方程式を求めると$\det(xI - A(\alpha)) = (x - 1)^3$である。固有値1に属する固有空間の次元を求めると
$$
\dim \ker(I - A(\alpha))= \begin{cases} 2 & (\alpha = 0, 1) \\ 1 & (\alpha \neq 0,1) \end{cases}
$$
であるから、$A(\alpha)$のJordan標準形は
$\alpha = 0,1$のとき$\begin{pmatrix} 1 &0 &0 \\ 0 &1&1 \\ 0&0 &1  \end{pmatrix} $\ \
$\alpha \neq 0,1$のとき$\begin{pmatrix} 1 &1&0 \\ 0 &1&1 \\ 0&0 &1  \end{pmatrix}$
である。したがって$A(\alpha)$と$A(\beta)$が相似であることと、$\alpha, \beta \in \{ 0,1 \}$または$\alpha, \beta \in \mathbb{R} \setminus \{ 0,1 \}$であることが同値。\\ \\

\section*{問2}
\noindent
$D =\{(x, y) \in \mathbb{R}^2 \mid 0 \leq y \leq 1, -\sqrt{y} \leq x \leq \sqrt{y} \}$であることから、
$$
\int_{D} |x|^3 e^{- y^3}dxdy =\int_{0}^{1} \int_{-\sqrt{y}}^{\sqrt{y}}  |x|^3 e^{- y^3}dxdy = \frac{1 - e^{-1}}{6}
$$ \\ \\

\section*{問3}
\noindent
(1)
$$
\ker \begin{pmatrix}1&0&1 \\ 0&1&1 \\ 1&0&0 \\ 0&1&0 \\ 1&0&1 \end{pmatrix} = 0
$$
より$\langle a_1, a_2, a_3 \rangle$は一次独立だから$\dim W_1 = 3$.
$$
\dim \ker \begin{pmatrix}1&0&1 \\ 1&1&0 \\ 0&0&0 \\ 1&1&0 \\ 1&0&1 \end{pmatrix} = 1
$$
より$\dim W_2 = 3 - 1 =2$である。\\
(2) $W_1 + W_2 = \langle a_1, a_2, a_3 , a_4 \rangle$であることに注意する。
$$
\ker \begin{pmatrix}1&0&1&1 \\ 0&1&1&1 \\ 1&0&0&0 \\ 0&1&0&1 \\ 1&0&1&1 \end{pmatrix} = 0
$$
より$\dim (W_1 + W_2) = 4$ \\
(3)
$$
c_1a_1 + c_2a_2 + c_3a_3 = c_4a_4 + c_5a_5 + c_6a_6
$$
とする。すると$\{ a_1, a_2, a_3 , a_4\}$の一次独立性から$c_1 = c_3 = 0$を得る。よって、$W_1 \cap W_2 = \langle a_2 \rangle $だから$\dim (W_1 \cap W_2) = 1$ \\ \\

\section*{問4}
\noindent
(1) $x = 1$で0に収束することから$\alpha > 1$を得る。逆に$\alpha > 1$なら、
$$
\lim_{n \to \infty} \frac{nx^2}{(1 + nx)^{\alpha}}
=\lim_{n \to \infty} \frac{n^{1-\alpha}x^{2 - \alpha}}{(1 + \frac{1}{(nx)^{\alpha}})^{\alpha}} = 0.
$$
よって求める条件は$\alpha > 1$ \\
(2) $\{ f_n\}$が区間$[0, \infty)$上で0に一様収束するなら自然数$m$について
$$
\| f_n\|_{\infty} = \sup_{x \in [0, \infty)} \frac{nx^2}{(1 + nx)^{\alpha}}
=\sup_{x \in [0, \infty)} \frac{n^{1-\alpha}x^{2 - \alpha}}{(1 + \frac{1}{(nx)^{\alpha}})^{\alpha}}
\geq \sup_{m \geq 1} \frac{n^{2m+1-(m+1)\alpha}}{(1+\frac{1}{(n^{m+1})^{\alpha}})^{\alpha}}
$$
の左辺が0に収束する。よって右辺も0に収束し、
$$
\lim_{n \to \infty} \sup_{m \geq 1} n^{2m+1-(m+1)\alpha} = 0
$$
でなければならない。よって$\forall m \geq 1 \ 2m+1-(m+1)\alpha < 0$. \ $m \rightarrow \infty$として$2 \leq \alpha $である。\\
逆に$2 \leq \alpha $なら
$$
\| f_n\|_{\infty} = \sup_{x \in [0, \infty)} \frac{1}{n} \frac{(nx)^2}{(1 + nx)^{2}} \frac{1}{(1+nx)^{\alpha - 2}}
\leq \frac{1}{n}
$$
より0に一様収束することがいえる。したがって求める条件は、$2 \leq \alpha $.

\newpage
\section*{2017年問題}
\section*{問1}
次の行列A, Bは対角化可能か判定せよ。理由も示せ。
\subsection*{(1) }
$
A = \left(
   \begin{array}{ccr}
    1 & 1 & -1 \\-4 & 6 & -7 \\-3 & 3 & -4
   \end{array}
     \right)
$
\\
\subsection*{(2)}
$
B = \begin{pmatrix}
    8 & -12 & 6 \\
    3 & -4 & 3 \\
    -3 & 6 & -1
    \end{pmatrix}
$   \\


\section*{問2}
\noindent
$M_{2}(\mathbb{R})$を実2次正方行列全体とする。$M_{2}(\mathbb{R})$は$\mathbb{R}$ベクトル空間である。\\
$N = \begin{pmatrix} 0 & 1 \\ 0& 0 \end{pmatrix}$とおく。$\Phi:M_{2}(\mathbb{R}) \rightarrow M_{2}(\mathbb{R})$を
$\Phi(X) = NX - XN$によって定める。\\ \\
(1) $\Phi$は線形写像であることを示せ。\\
(2) $M_{2}(\mathbb{R})$の基底$\{ e_1, \ldots ,e_4 \}$を
$$
e_1 = \begin{pmatrix} 1&0\\0&0 \end{pmatrix} \hspace{10pt}
e_2 = \begin{pmatrix} 0&1\\0&0 \end{pmatrix} \hspace{10pt}
e_3 = \begin{pmatrix} 0&0\\1&0 \end{pmatrix} \hspace{10pt}
e_4 = \begin{pmatrix} 0&0\\0&1 \end{pmatrix} \hspace{10pt}
$$
と定める。ここで$\Phi ( e_{j} ) = \sum_{i = 1}^{4} c_{ij}e_{i} $とするとき、行列$C = (c_{ij})_{1 \leq i,j \leq 4}$を求めよ。\\
(3) $\Phi$の階数を求め、Ker$\Phi$の基底をひとつ与えよ。\\

\section*{問3}
$D = \{ (x, y) \in \mathbb{R}^2 \mid 0 <(x^2 + y^2)^{ \frac{3}{2} } \leq x \} $なるとき、重積分
$$
\int_D \exp( \frac{y}{ \sqrt{x^2 + y^2} } ) dxdy
$$
の値を求めよ。\\

\section*{問4}
$\alpha > 1$なる$\alpha \in \mathbb{R}$を固定する。\\ \\
(1) 任意の$t \in \mathbb{R}$について広義積分
$$
F(t) = \int_0^{\infty} \frac{ \tan^{-1}(tx) }{x(1+x^{\alpha} )} dx
$$
が存在することを示せ。ただし$\tan^{-1}$は$\tan$を$(-\frac{\pi}{2}, \frac{\pi}{2})$へ制限したものの逆写像である。\\
(2) $F(t)$は$\mathbb{R}$上一様連続であることを示せ。\\






\newpage
\section*{2017年解答例}

\section*{問1}
\noindent
(1) 特性方程式を求めると
$$
\det(xI - A) = (x + 1)(x - 2)^2
$$
を得る。固有値2に属する固有空間を求めると
$$
\ker(2I - A) = \left \{ x \begin{pmatrix} 1 \\ 1 \\ 0 \end{pmatrix} \in \mathbb{R}^3 \mid x \in \mathbb{R} \right \}
$$
であり、固有空間の直和の次元は2だから、$\mathbb{R}^3 $全体にはならない。よって$A$は対角化可能でない。\\
実際、$P = \begin{pmatrix} 0&1&-2 \\ 1&1&0 \\ 1&0&1 \end{pmatrix}$とすると$P^{-1}AP$はJordan標準形$\begin{pmatrix} -1&0&0 \\ 0&2&1 \\ 0&0&2 \end{pmatrix}$である。\\ \\

\noindent
(2) 特性方程式を求めると
$$
\det(xI - B) = (x + 1)(x - 2)^2
$$
を得る。固有値2に属する固有空間を求めると
$$
\ker(2I - B) = \ker \begin{pmatrix} 1&-2&1 \\ 0&0&0 \\ 0&0&0 \end{pmatrix}
$$
であり、$\dim (\ker (2I - B)) = 2$である。よって$\mathbb{R}^3$全体が固有空間の直和に等しく, $B$は対角化可能と判る。\\
実際$P = \begin{pmatrix} 2&2&-1 \\ 1&1&0 \\ -1&0&1 \end{pmatrix}$とおくと$P^{-1} BP$は対角行列である。\\ \\

\section*{問2}
\noindent
(1) $\Phi$がスカラー倍と和を保つことから判る。\\
(2) (i. j)成分だけが1で、あとは0という行列を$E_{ij}$で表すことにする。このとき
$$
E_{ij}E_{lk} = \delta_{jl}E_{ik}
$$
が成り立つ。(ただし$\delta_{jl}$はKroneckerのデルタである) したがって
$$
\Phi(E_{ij}) = \delta_{2i}E_{1j} - \delta_{j1}E_{i2}
$$
と計算できるので、代入して
$$
\Phi(e_1)  =  - e_2  \ \ \
\Phi(e_2)  =  0  \ \ \
\Phi(e_3)  =  e_1 - e_4 \ \ \
\Phi(e_4)  =  e_2
$$
ゆえに
$$
C  = \begin{pmatrix} 0&-1&0&0 \\ 0&0&0&0 \\ 1&0&0&-1 \\ 0&1&0&0 \end{pmatrix}^{\mathrm{T}} 
 =  \begin{pmatrix} 0&0&1&0 \\ -1&0&0&1 \\ 0&0&0&0 \\ 0&0&-1&0 \end{pmatrix}
$$
\\ \\

\noindent
(3) $\ker C$の基底として$\begin{pmatrix} 1 \\ 0 \\ 0 \\ 1 \end{pmatrix}$、$\begin{pmatrix} 0 \\1 \\ 0 \\ 0 \end{pmatrix}$がとれる。よって$\ker \Phi$の基底として$\{e_1 +e_4, e_2 \}$がとれ、$\Phi$の階数は$4 - 2 = 2$である。\\


\section*{問3}
\noindent
極座標変換
$$
\begin{cases}
x = r \cos \theta \\ y = r \sin \theta \end{cases} \ \ \
\frac{\partial (x, y)}{\partial (r, \theta)} = r
$$
により、$E = \{(r, \theta) \in \mathbb{R}^2 \mid 0 < r^3 \leq r \cos \theta , |\theta| \leq \pi \}$とすると
$$
\int_D \exp( \frac{y}{ \sqrt{x^2 + y^2} } ) dxdy = \int_E r \exp(\sin \theta)drd\theta
$$
が成り立つ。ここで$E = \{(r, \theta) \in \mathbb{R}^2 \mid 0 <  r \leq \sqrt{\cos \theta} , \ |\theta| \leq \frac{\pi}{2} \}$だから
$$
\int_E r \exp(\sin \theta)drd\theta
 = \int_{- \frac{\pi}{2}}^{\frac{\pi}{2}} \int_{0}^{\sqrt{\cos \theta}}  r \exp (\sin\theta)drd\theta
 =  \frac{e - e^{-1}}{2}
$$ \\ \\

\section*{問4}
\noindent
(1) $\phi(x) = \tan^{-1} (x)$とする。$0 \leq \phi’(x) \leq 1$であることに注意すると
Taylorの定理により
$$
\left| \frac{\phi(tx)}{x} \right| =\left| \frac{\phi(tx) - \phi(0)}{tx - 0} \right| \cdot |t| \leq |t|
$$
がわかる。
したがって
$$
\int_{0}^{\infty} \left| \frac{\phi(tx)}{x(1 + x^{\alpha})} \right| dx
\leq |t| \int_{0}^{\infty} \frac{dx}{1 + x^{\alpha}}
$$
が成り立つ。ここで
$$
\int_{0}^{\infty} \frac{dx}{1 + x^{\alpha}}
\leq  \int_{0}^{1} \frac{dx}{1 + x^{\alpha}} +  \int_{1}^{\infty} \frac{dx}{1 + x^{\alpha}}
\leq \frac{\alpha}{\alpha - 1}
$$
であるから
$$
\int_{0}^{\infty} \left| \frac{\phi(tx)}{x(1 + x^{\alpha})} \right| dx
\leq \frac{\alpha}{\alpha - 1} |t|
$$
であることが判る。\\

\noindent
(2) $t, s \in \mathbb{R}$	とする。 (1)と同様にして
$$
|F(t) - F(s)|
\leq \frac{\alpha}{\alpha - 1} |t - s|
$$
であり、$F$はLipschitz連続、とくに一様連続である。



\end{document}